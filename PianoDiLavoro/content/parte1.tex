\section{Contatti}
\textbf{Studente}: \myName, \email{valentina.signor@studenti.unipd.it}, + 39 342 083 14 26;\\
\textbf{Tutor aziendale}: \Greg, \email{gregorio.piccoli@zucchetti.it}, + 39 0371 59 457 11;\\
\textbf{Azienda}: \myCompany, Via Giovanni Cittadella 7, 735137 - Padova, \sitoCompany.
soluzioni software, hardware e servizi. 

\section{Scopo dello stage}
\textit{\myCompany}, con la sua esperienza di oltre 40 anni rappresenta la prima software house italiana per storia e dimensione. Fondata nel 1978 ad opera di Domenico Zucchetti, per commercializzare in tutta Italia un software che garantisse la gestione automatizzata della dichiarazione dei redditti, ad oggi non solo permette ad Aziende, Professionisti, Associazioni di Categoria e Pubblica Amministrazione di disporre di  soluzioni software, hardware e servizi innovativi, ma si è diffusa anche all'estero presentando sedi in Francia, Germania, Romania, Spagna, Svizzera, Brasile e Stati Uniti.\\ Nell'innovativa offerta che \textit{\myCompany{}} mette a disposizione dei propri clienti si possono identificare una vasta gamma di prodotti operanti negli ambiti:
\begin{itemize}
\item gestionali: contabilità, acquisti, vendite, magazzino ecc.;
\item delle Risorse umane: paghe, stipendi, presenze, controllo accessi ecc.;
\item fiscali: dichiarazione dei redditi, 730, conservazione sostitutiva ecc.
\end{itemize}
Oltre che molti altri prodotti di contorno nei campi della Business Intelligence, robotica, IoT e  sicurezza.
\\\\
Lo stage prevede l'inserimento dello studente nella realtà aziendale al fine di garantirne la formazione attraverso compiti reali e sperimentazione diretta dei processi e contesti lavorativi. \\
Il progetto da sviluppare permetterà di ricavare a partire dall'osservazione di eventi reali, le conseguenti \textit{Leggi empiriche} e le teorie caratterizzanti. Quest'ultime permetteranno calate nell'ambito degli algoritmi di Machine Learning, di fornire delle solide basi ai dati osservati in modo da identificare un pattern comportamentale che li spieghi. 
\\I passi a cui lo studente dovrà attenersi saranno i seguenti:
\begin{itemize}
\item a seguito dell'osservazione di eventi reali, si formuleranno Leggi empiriche e Teorie;
\item descrizione delle varie fasi di analisi del dato; essa si articolerà attraverso preparazione e pulizia del dato, analisi preliminare, esposizione dei dati, analisi predittiva e analisi prescrittiva;  
\item adattamento e applicazione di algoritmi di Supervised Learning: Regressione Lineare, SVM e Random Forest;
\item adattamento e applicazione di algoritmi di Unsupervised Learning: k-Means Clustering;
\item applicazione delle Reti Bayesiane al fine di stabilire delle leggi;
\item introduzione di algoritmi di predizione per poter predire il comportamento dei dati e le leggi che li caratterizzano;
\item riepilogo critico e documentato, attraverso anche un'attività di testing, dei risultati ottenuti.
\end{itemize}

\section{Interazione tra studente e tutor aziendale}
Vista la situazione particolare in cui verte il nostro Paese da inizio marzo 2020 (causa COVID-19), si assicura che lo stage in caso di necessità possa essere svolto anche in modalità smart-working. In tal caso le regole a cui attenersi per verificare lo stato di avanzamento del lavoro, chiarirne eventualmente gli obiettivi e aggiornare il Piano di Lavoro se necessario, saranno le seguenti:
\begin{itemize}
\item lo studente si terrà comunque in stretta comunicazione col proprio tutor aziendale \Greg; ciò dovrà avvenire almeno quotidianamente, attraverso attività di video‐conferenza o tramite altri canali digitali;
\item sarà necessario un riconoscimento dell'impegno orario dallo studente; quest'ultimo provvederà quindi a redigere un registro, informale ma preciso, che giornalmente verrà sottoposto a convalida dal tutor aziendale. Sarà proprio tale registro ad abilitare la firma del tutor aziendale sul modulo di fine stage, dando testimonianza dello svolgimento delle ore di lavoro richieste.
\end{itemize}  
Nel caso in cui lo stage potesse essere svolto in presenza lo studente farà sempre riferimento al tutor aziendale \Greg{} con il quale lavorerà a stretto contatto. \\Tale strategia si auspica di facilitare l'integrazione dello stagista all'interno dell'ambiente lavorativo adeguandolo alle caratteristiche del periodo corrente.

\section{Contenuti formativi previsti}
Durante questo progetto di stage lo studente avrà occasione di approfondire le sue conoscenze:
\begin{itemize}
\item nell'osservazione e analisi critica dei dati attraverso:
	\begin{itemize}
	\item osservazione di fenomeni reali al fine di ricavarne regole precise che ne spieghino il comportamento;
	\item l'uso del software Orange Canvas;
	\item tecniche di pulizia del dato.
	\end{itemize}
\item nelle aree del Machine Learning attraverso:
	\begin{itemize}
	\item lo studio, l'adattamento e l'applicazione di differenti algoritmi di analisi e predizione sui dati;
	\item utilizzo di tecniche di Supervided ed Unsupervised Learning.
	\end{itemize}
\end{itemize} 





 









