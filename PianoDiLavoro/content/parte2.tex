\section{Pianificazione del Lavoro}
\subsection{Pianificazione settimanale}
\begin{itemize}
\item \textbf{Prima Settimana - Studio del problema attraverso Osservazioni, Leggi empiriche e Teorie};
	\begin{itemize}
	\item preparazione e analisi dei dati;
	\item studio del software Orange Canvas.
	\end{itemize}
	Durante questa settimana lo studente impiegherà parte del monte delle ore previste ad inquadrare il problema, predisponendo il lavoro che si articolerà  nelle settimane successive. Si approccerà al software Orange Canvas studiandone il funzionamento, facendone uso per facilitare almeno inizialmente l'attività di analisi dei dati. In particolare si dedicherà alla  preparazione e alla pulizia dei valori osservati, per poi passare alla loro esposizione e analisi (preliminare, predittiva e prescrittiva). Potrà inoltre osservare i comportamenti dei vari algoritmi di Machine Learning confrontandone le performance e la precisione. Lo studente sarà quindi chiamato ad effettuare uno studio approfondito circa le modalità di osservazione e manipolazione dei dati in modo da localizzarne le conseguenti regole empiriche e teorie caratterizzanti.

\item \textbf{Seconda Settimana - Programma per la Regressione Lineare in Javascript, Java e SQL};
	\begin{itemize}
	\item studio dell'algoritmo di Regressione Lineare;
	\item studio e adattamento della libreria corrispondente;
	\item aggiunta previsione in Java;
	\item aggiunta previsione in SQL.
	\end{itemize}
	Durante questa settimana lo studente si impegnerà a prendere confidenza con l'algoritmo di Regressione Lineare, anche grazie all'utilizzo di Orange Canvas; successivamente procederà all'adattamento della relativa libreria Javascript fornita dall'azienda: l'obiettivo è quello di  renderla omogenea con le altre librerie "sorelle" relative agli altri algoritmi di Machine Learning. Lo studente cercherà quindi di creare per quanto possibile, un'interfaccia comune mettendo mano dove abbia senso, alla struttura, ai metodi e alle variabili della libreria in modo da ottenere un risultato che sia il più manutenibile possibile. Provvederà poi adottando il linguaggi Java ed SQL, ad estendere tale libreria aggiungendovi la possibilità di fare previsioni. 

\item \textbf{Terza Settimana - Programma per la Support Vector Machines in Javascript, Java e SQL};
	\begin{itemize}
	\item studio dell'algoritmo di Support Vector Machines;
	\item studio e adattamento della libreria corrispondente;
	\item aggiunta previsione in Java; 
	\item aggiunta previsione in SQL.
	\end{itemize}
	Durante questa settimana lo studente si impegnerà a prendere confidenza con l'algoritmo di Support Vector Machines anche aiutandosi con Orange Canvas; successivamente procederà come trattato al punto sopra all'adattamento della relativa libreria Javascript ed alla sua estensione attraverso l'aggiunta del algoritmo di previsione.


\item \textbf{Quarta Settimana - Test e Documentazione.}
	\par Durante questa settimana, lo stagista documenterà e testerà la bontà del codice da lui prodotto durante le settimane precedenti.

\item \textbf{Quinta Settimana - Programma per k-Means in Javascript, Java e SQL};
\begin{itemize}
	\item studio dell'algoritmo non supervisionato k-Means;
	\item studio e adattamento della libreria corrispondente;
	\item aggiunta previsione in Java;
	\item aggiunta previsione in SQL.
	\end{itemize}
	Durante questa settimana lo studente svolgerà uno studio individuale sugli algoritmi non supervisionati ponendo particolare attenzione al k-Means. Cercherà di interfacciarvisi grazie anche all'ausilio del software Orange Canvas, procedendo come fatto anche per gli algoritmi supervisionati, all'adattamento della relativa libreria Javascript ed alla sua estensione attraverso l'aggiunta del algoritmo di previsione.
	
	
\item \textbf{Sesta Settimana - Programma per Random Forest in Javascript, Java e SQL};
\begin{itemize}
	\item studio delle Random Forest;
	\item studio e adattamento della libreria corrispondente;
	\item aggiunta previsione in Java;
	\item aggiunta previsione in SQL.
	\end{itemize}
	Durante questa settimana lo studente passerà allo studio delle Random Forest, ne esaminerà le caratteristiche e il comportamento grazie anche all'uso di Orange Canvas; successivamente procederà come fatto per gli altri algoritmi, all'adattamento della relativa libreria Javascript fornita dall'azienda, ed alla sua estensione attraverso l'aggiunta del algoritmo di previsione.
	
	\item \textbf{Settima Settimana - Programma per Reti Bayesiane in Javascript, Java e SQL};
\begin{itemize}
	\item studio delle Reti Bayesiane;
	\item studio e adattamento della libreria corrispondente;
	\item aggiunta previsione in Java;
	\item aggiunta previsione in SQL.
	\end{itemize}
	Durante questa settimana lo studente si impegnerà a prendere confidenza con le Reti Bayesiane, studiandone le caratteristiche e il comportamento grazie anche all'uso di Orange Canvas. Con esse lo studente potrà localizzare delle teorie che gli permetteranno di spiegare attraverso una base ragionata i dati osservati inizialmente. Successivamente procederà come fatto per gli atri algoritmi all'adattamento della relativa libreria Javascript ed alla sua estensione attraverso l'aggiunta del algoritmo di previsione.
	
	\item \textbf{Ottava Settimana - Test e Documentazione}.
	 \par Durante questa settimana, lo stagista come fatto anche dopo le prime tre settimane di lavoro, provvederà a documentare e testare la bontà del codice da lui prodotto nelle settimane passate.

\end{itemize}