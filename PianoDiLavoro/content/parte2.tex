\section{Pianificazione del Lavoro}
\subsection{Pianificazione settimanale}
\begin{itemize}
\item \textbf{Prima Settimana - Osservazioni, Leggi empiriche e Teorie: descrizioni delle varia fasi dell’analisi dei dati}
	\begin{itemize}
	\item preparazione e pulizia del dato;
	\item varie tecniche di analisi del dato.
	\end{itemize}

	Durante questa settimana lo studente impiegherà parte del monte delle ore previste a svolgere attività di preparazione e pulizia del dato, analisi preliminare, esposizione dei dati, analisi predittiva e analisi prescrittiva. Lo studente sarà
chiamato ad effettuare uno studio approfondito circa le modalità di osservazione e manipolazione dei dati in modo da locanizzarne le conseguenti regole empiriche caratterizzanti.

\item \textbf{Seconda Settimana - Programma per la Regressione Lineare in Javascript, Java e SQL}

\item \textbf{Terza Settimana - Programma per la Support Vector Machines in Javascript, Java e SQL}


\item \textbf{Quarta Settimana - Test e Documentazione}
	\begin{itemize}
	\item Durante questa settimana, lo stagista documenterà e testerà la bontà del codice da lui prodotto durante le settimane precedenti.
	\end{itemize}





\end{itemize}