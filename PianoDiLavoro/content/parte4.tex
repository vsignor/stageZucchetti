\section{Obiettivi}
\subsection{Notazione}
Si farà riferimento ai requisiti seguendo le seguenti notazioni:
\begin{itemize}
\item \textbf{O}: utilizzato per i \textit{requisiti obbligatori}, sono da considerarsi vincolanti in quanto obiettivo primario richiesto dal committente;
\item \textbf{D}: utilizzato per i requisiti \textit{desiderabili}, sono da considerarsi non vincolanti o strettamente necessari, ma dal riconoscibile valore
aggiunto;
\item \textbf{F}: utilizzato per i \textit{requisiti facoltativi}, sono da considerarsi portatori di valore aggiunto non strettamente competitivo.
\end{itemize}
Le sigle precedentemente indicate saranno seguite da una coppia sequenziale di numeri, identificativo del
requisito.

\pagebreak

\section{Prodotti attesi a fine stage}