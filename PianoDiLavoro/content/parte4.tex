\section{Obiettivi}
\subsection{Notazione}
Si farà riferimento ai requisiti seguendo le seguenti notazioni:
\begin{itemize}
\item \textbf{O}: utilizzato per i \textit{requisiti obbligatori}, sono da considerarsi vincolanti in quanto obiettivo primario richiesto dal committente;
\item \textbf{D}: utilizzato per i requisiti \textit{desiderabili}, sono da considerarsi non vincolanti o strettamente necessari, ma dal riconoscibile valore
aggiunto;
\item \textbf{F}: utilizzato per i \textit{requisiti facoltativi}, sono da considerarsi portatori di valore aggiunto non strettamente competitivo.
\end{itemize}
Le sigle precedentemente indicate saranno seguite da una coppia sequenziale di numeri, rappresentati l'identificativo del requisito.

{\renewcommand{\arraystretch}{2}
\begin{longtable}{|p{1cm}| p{12.25cm} |}
	% ----- Intestazione ----- %
	\hline
	\rowcolor{blue} \multicolumn{2}{|c|}{
	\textbf{Requisiti Obbligatori}
	} \\
	% ------------------------ 
		\endhead
	% ------------------------ % 		
		\hline \rowcolor{lightbrown}
		O01 & 
		Individuazione e analisi di Leggi empiriche  e Teorie. \\	
	% ------------------------ %
		\hline \rowcolor{lighterbrown}
		O02 & 
		Studio degli algoritmi di Regressione Lineare e Support Vector Machine. \\	
		% ------------------------ % 		
		\hline \rowcolor{lightbrown}
		O03 & 
		Studio degli algoritmi di k-Means, Random Forest e Reti Bayesiane.\\	
	% ------------------------ %
	\hline \rowcolor{lighterbrown}
		O04 & 
		Creazione di un'interfaccia comune per gli algoritmi di Machine Learning analizzati.\\	
	% ------------------------ %
	\hline \rowcolor{lightbrown}
		O05 & 
		Implementazione previsione in Java.\\	
	% ------------------------ %
	\hline \rowcolor{lighterbrown}
		O06 & 
		Produzione di una documentazione completa come resoconto delle attività svolte e delle
osservazioni effettuate.\\	
	% ------------------------ %
	\hline \rowcolor{lighterbrown}
		O07 & 
		Mantenimento della riservatezza necessaria nei confronti dei materiali forniti e dei dati osservati durante lo svolgimento dello stage.\\	
	% ------------------------ %
	\hline
\end{longtable}}

{\renewcommand{\arraystretch}{2}
\begin{longtable}{|p{1cm}| p{12.25cm} |}
	% ----- Intestazione ----- %
	\hline
	\rowcolor{blue} \multicolumn{2}{|c|}{
	\textbf{Requisiti Desiderabili}
	} \\
	% ------------------------ 
		\endhead
	% ------------------------ % 		
		\hline \rowcolor{lightbrown}
		D01 & 
		 Per ogni algoritmo la previsione in SQL.\\	
	% ------------------------ %
	\hline
\end{longtable}}

{\renewcommand{\arraystretch}{2}
\begin{longtable}{|p{1cm}| p{12.25cm} |}
	% ----- Intestazione ----- %
	\hline
	\rowcolor{blue} \multicolumn{2}{|c|}{
	\textbf{Requisiti Fcoltativi}
	} \\
	% ------------------------ 
		\endhead
	% ------------------------ % 		
		\hline \rowcolor{lightbrown}
		F01 & 
		Reti Bayesiane + storia documentazione. \\	
	% ------------------------ %
	\hline
\end{longtable}}

\pagebreak

\section{Prodotti attesi a fine stage}