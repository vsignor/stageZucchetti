\section{Obiettivi fissati}
\subsection{Notazione}
Si farà riferimento ai requisiti seguendo le seguenti notazioni:
\begin{itemize}
\item \textbf{O}: requisito \textit{Obbligatorio}, da considerarsi vincolante in quanto obiettivo primario richiesto dal committente;
\item \textbf{D}: requisito \textit{Desiderabile}, non vincolante o strettamente necessario, ma dal riconoscibile valore aggiunto;
\item \textbf{F}: requisito \textit{Facoltativo}, la cui implementazione renderebbe il sistema più completo, portando,
con molta probabilità, ad un dispendio di risorse con conseguente aumento dei
costi.
\end{itemize}
Le sigle sopra esposte saranno seguite da una coppia sequenziale di numeri, rappresentanti il \textit{Codice} del requisito.

{\renewcommand{\arraystretch}{2}
\begin{longtable}{|p{1cm}| p{12.25cm} |}
	% ----- Intestazione ----- %
	\hline
	\rowcolor{blue} \multicolumn{2}{|c|}{
	\textbf{\textcolor{white}{Obbligatori}}
	} \\
	% ------------------------ 
		\endhead
	% ------------------------ % 		
		\hline \rowcolor{lightbrown}
		O01 & 
		Individuazione e analisi di Leggi empiriche  e Teorie. \\	
	% ------------------------ %
		\hline \rowcolor{lighterbrown}
		O02 & 
		Studio degli algoritmi di Regressione Lineare e Support Vector Machine. \\	
		% ------------------------ % 		
		\hline \rowcolor{lightbrown}
		O03 & 
		Studio degli algoritmi di k-Means e Random Forest.\\	
	% ------------------------ %
	\hline \rowcolor{lighterbrown}
		O04 & 
		Adattamento delle varie librerie in modo da renderle uniformi.\\	
	% ------------------------ %
	\hline \rowcolor{lightbrown}
		O05 & 
		Creazione di un'interfaccia comune per gli algoritmi di Machine Learning analizzati.\\	
	% ------------------------ %
	\hline \rowcolor{lighterbrown}
		O06 & 
		Implementazione previsione in Java.\\	
	% ------------------------ %
	\hline \rowcolor{lightbrown}
		O07 & 
		Produzione di una documentazione completa come resoconto dei test, delle attività svolte e delle osservazioni fatte.\\	
	% ------------------------ %
	\hline \rowcolor{lighterbrown}
		O07 & 
		Mantenimento della riservatezza necessaria in merito ai materiali forniti e ai dati osservati durante lo svolgimento dello stage.\\	
	% ------------------------ %
	\hline
	\caption{Obiettivi Obbligatori}\label{tab:obb-ob}
\end{longtable}}

{\renewcommand{\arraystretch}{2}
\begin{longtable}{|p{1cm}| p{12.25cm} |}
	% ----- Intestazione ----- %
	\hline
	\rowcolor{blue} \multicolumn{2}{|c|}{
	{\textcolor{white}{Desiderabili}}
	} \\
	% ------------------------ 
		\endhead
	% ------------------------ % 		
		\hline \rowcolor{lightbrown}
		D01 & 
		Implementazione per ogni algoritmo in esame della previsione in SQL.\\	
	% ------------------------ %
	\hline
	\caption{Obiettivi Desiderabili}\label{tab:des-ob}
\end{longtable}}

{\renewcommand{\arraystretch}{2}
\begin{longtable}{|p{1cm}| p{12.25cm} |}
	% ----- Intestazione ----- %
	\hline
	\rowcolor{blue} \multicolumn{2}{|c|}{
	{\textcolor{white}{Facoltativi}}
	} \\
	% ------------------------ 
		\endhead
	% ------------------------ % 		
		\hline \rowcolor{lightbrown}
		F01 & 
		Studio delle Reti Bayesiane con adattamento della libreria corrispondente.\\	
	% ------------------------ %
		\hline \rowcolor{lighterbrown}
		F02 & 
		Aggiunta previsione in Java nel contesto delle Reti Bayesiane. \\	
	%------------------------ %
	\hline \rowcolor{lightbrown}
		F03 & 
		Aggiunta previsione in SQL nel contesto delle Reti Bayesiane. \\	
	% ------------------------ %
	\hline \rowcolor{lighterbrown}
		F04 & 
	Produzione della documentazione contenete il resoconto di quanto fatto e i risultati ottenuti in merito alla Reti Bayesiane. \\
	% ------------------------ %
	\hline
	\caption{Obiettivi Facoltativi}\label{tab:fac-ob}
\end{longtable}}

\pagebreak

\section{Prodotti attesi a fine stage}
 Di seguito vengono esposti i prodotti che si attendono dallo stage.
{\renewcommand{\arraystretch}{2}
\begin{longtable}{|p{13.25cm}|}
	% ----- Intestazione ----- %
	\hline
	\rowcolor{blue} \multicolumn{1}{|c|}{
	\textbf{\textcolor{white}{Obbligatori}}
	} \\
	% ------------------------ 
		\endhead
	% ------------------------ % 
		\hline \rowcolor{lightbrown}
		 Uniformare le varie librerie relative agli algoritmi di Machine Learning.\\	
	% ------------------------ % 
		\hline \rowcolor{lighterbrown}
		 Creazione di un'interfaccia comune in JavaScript per gli algoritmi analizzati.\\	
	% ------------------------ % 
		\hline \rowcolor{lightbrown}
		 Implementazione della funzionalità di previsione in Java. \\	
	% ------------------------ %
	\hline \rowcolor{lighterbrown}
		Produzione della documentazione relativa all'interfaccia realizzata e testing. \\	
	% ------------------------ %
	\hline
	\caption{Prodotti Obbligatori}\label{tab:obb-prod}
\end{longtable}}

{\renewcommand{\arraystretch}{2}
\begin{longtable}{|p{13.25cm}|}
	% ----- Intestazione ----- %
	\hline
	\rowcolor{blue} \multicolumn{1}{|c|}{
	\textbf{\textcolor{white}{Desiderabili}}
	} \\
	% ------------------------ 
		\endhead
	% ------------------------ % 	
	\hline \rowcolor{lightbrown}
		 Implementazione della funzionalità di previsione in SQL. \\	
	% ------------------------ %
	\hline
	\caption{Prodotti Desiderabili}\label{tab:des-prod}
\end{longtable}}

{\renewcommand{\arraystretch}{2}
\begin{longtable}{|p{13.25cm}|}
	% ----- Intestazione ----- %
	\hline
	\rowcolor{blue} \multicolumn{1}{|c|}{
	\textbf{\textcolor{white}{Facoltativi}}
	} \\
	% ------------------------ 
		\endhead
	% ------------------------ % 	
		\hline \rowcolor{lightbrown}
		 Adattamento della libreria riguardante le Reti Bayesiane.\\	
	% ------------------------ %
	\hline \rowcolor{lighterbrown}
		 Predizione in Java per le Reti Bayesiane.\\	
	% ------------------------ %
	\hline \rowcolor{lightbrown}
		 Predizione in SQL per le Reti Bayesiane.\\	
	% ------------------------ %
	\hline \rowcolor{lighterbrown}
		 Produzione della documentazione relativa e test su quanto prodotti nell'ambito delle Reti Bayesiane.\\	
	% ------------------------ %
	\hline
	\caption{Prodotti Facoltativi}\label{tab:fac-prod}
\end{longtable}}


