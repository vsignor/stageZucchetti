\documentclass[10pt,a4paper]{article}
%**************************************************************
% Importazione package
%************************************************************** 

\usepackage[T1]{fontenc}                % codifica dei font:
                                        % NOTA BENE! richiede una distribuzione *completa* di LaTeX

\usepackage[utf8]{inputenc}             % codifica di input; anche [latin1] va bene
                                        % NOTA BENE! va accordata con le preferenze dell'editor

\usepackage[english, italian]{babel}    % per scrivere in italiano e in inglese;
                                        % l'ultima lingua (l'italiano) risulta predefinita
                                        
\usepackage{bookmark} 

\usepackage[a4paper,left=2cm,right=2cm, top=2.5cm, bottom=3cm]{geometry}                  % segnalibri

\usepackage{caption}                    % didascalie

\usepackage{chngpage,calc}              % centra il frontespizio

\usepackage{csquotes}                   % gestisce automaticamente i caratteri (")

\usepackage{emptypage}                  % pagine vuote senza testatina e piede di pagina

\usepackage{epigraph}			% per epigrafi

\usepackage{eurosym}                    % simbolo dell'euro

%\usepackage{indentfirst}               % rientra il primo paragrafo di ogni sezione

\usepackage{graphicx}                   % immagini

\usepackage{hyperref}                   % collegamenti ipertestuali

\usepackage{listings}                   % codici

\usepackage{microtype}                  % microtipografia

\usepackage{mparhack,fixltx2e,relsize}  % finezze tipografiche

\usepackage{nameref}                    % visualizza nome dei riferimenti                                      

\usepackage[font=small]{quoting}        % citazioni

\usepackage{subfig}                     % sottofigure, sottotabelle

\usepackage[italian]{varioref}          % riferimenti completi della pagina

\usepackage[dvipsnames]{xcolor}         % colori
\usepackage{textcomp}										% per inserire il simbolo di grado

\usepackage{booktabs}                   % tabelle                                       
\usepackage{tabularx}                   % tabelle di larghezza prefissata                                    
\usepackage{longtable}                  % tabelle su più pagine                                        
\usepackage{ltxtable}                   % tabelle su più pagine e adattabili in larghezza

                                        
\usepackage{float}											% permette di non far posizionare le immagini alla pagina seguete
\usepackage{multirow}
\usepackage{fancyhdr} % per header e footer

\usepackage{hyperref} % link
\definecolor{linkcolor}{cmyk}{1,.60,0,.40}
\definecolor{acqua}{rgb}{0.0, 1.0, 1.0}
\definecolor{blizzardblue}{rgb}{0.67, 0.9, 0.93}
\definecolor{lightcyan}{rgb}{0.88, 1.0, 1.0}
\definecolor{lightbrown}{rgb}{0.78, 0.7, 0.6}
\definecolor{lighterbrown}{rgb}{0.93, 0.86, 0.79}
\definecolor{brown}{rgb}{0.65, 0.49, 0.32}
\usepackage{hyperref}
\hypersetup{
	colorlinks=true,
	linkcolor=black,
	urlcolor=linkcolor
	}		

\usepackage{mathptmx} % font